\section{Conclusions}
With our small portable single turn coil apparatus, we were able to achieve 
fields in excess of 2\,T. We did some standard measurements of peak field 
in function of the initial voltage over the capacitor and the coil material 
and geometry.

We could not, however, find any significant correlation in the limited 
amount of tests we have done. A great deal more measurements need to be 
repeated in order to confidently show a proper relation between these 
variables, for which we did not have the time.

Secondly, the added measurement capabilities showed some difficulties when 
working with high voltage, high frequency signals. Choosing an appropriate 
voltage divider isn't a trivial task and measuring on such low timescales 
leads to uncertainties in the phases of different measurements. Care needs 
to be taken to minimize and take into account any parasitic impedances in 
the measuremnt circuit in order to still have some confidence of the 
measured amplitudes and phases. In our case, we suffered from the effect of 
the impedances of the connecting coax wires and the inductance of our shunt 
that causes significant non-linear effects, meaning that our current 
measurements are only (roughly) qualitatively correct.

Using these new measurement capabilities, we could (qualitatively) show 
some interesting results such as the (neglectible) single turn coil 
inductance and the plasma conduction and power dissipation after the coil 
desintegrates. Using better measurements, it would also be possible to get 
a rough estimate of the coil deformation as a result of the Maxwell stress 
by measuring the power and the magnetic field. Given the limited accuracy 
and reliabelity of our measurements, this was out of our scope.

% vim: ft=tex
