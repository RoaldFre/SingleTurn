\section{Results}
After having adapted the apparatus with the extra measuring capabilities, 
we performed some standard measurements of the maximum magnetic field. We also 
used the added $V$ and $I$ readings in a failed attempt to measure the power.

Because modifying the apparatus took up quite a bit of time, the 
measurements below are not as thorough as we preferred. We chose to touch 
several small topics that can be investigated further in the future.

\subsection{Power dissipated in the coil}

Using the values calibrated in the previous section, one can easily calculate 
the power dissipated or generated in the coil by multiplying them ($P = VI$).  
Doing this presents us with a plot like figure \ref{power}

Although at first sight, these results look sensible, when we integrate the 
power over time to find the total dissipated energy, we find that it is a 
multiple of the energy stored in the capacitor! As this is physically 
impossible, we have to conclude that either the voltage over or the current 
through the coil is measured incorrectly. Since the voltage is measured 
directly, the current is probably not completely right. Our experience showed 
that shifting the cables around a bit often could change the phase of the 
measured signal varies a considerable amount. Another problem is that the shunt 
used consists of a piece of wire that has some unnegligible self-inductance, 
meaning that it is not a linear component and thus can't be modeled as such.

Although we can only determine some qualitative behaviour, we're sure that if a 
method is used to determine the current with some degree of confidence, the 
power can certainly be determined as well.

\figOctave{power}{Power dissipated in the coil and plasma. Incorrect because 
more energy is dissipated than is available in the capacitor}

\subsection{Magnetic field in function of the width of the coil}
One expects to find an optimal value of the coil width. Too narrow a coil 
will experience a lot of Joule heating because of its high resistance. This 
can cause the metal to melt and evaporate before the maximum field strength 
is reached. A very wide coil will distribute its magnetic energy over a 
large volume, thus loosing efficiency.

The magnetic field of 28 measurements at maximum voltage was recorded. The 
results of these measurements are displayed in figure \ref{BvasteV}. The 
magnetic field is rendered in function of the width of the coil. Although 
we expected there to be an optimal coilwidth, we can discern no correlation 
between the field strength and the dimensions of the generating coil. We 
presume more measurements have to be performed under more strict conditions 
to determine any relation.

\figOctave{BvasteV}{Magnetic field in function of the coildwidth}

\subsection{Magnetic field in function of the bank voltage}

We also resarched the relationship between peak field and voltage of the 
capacitor bank. Measurements with a coil width of 6$\pm$1\,cm were selected and 
rendered in figure \ref{BvasteD}. The peak magnetic field is displayed in 
function of the voltage of the capacitor. Although no real relationship can be 
determined, it is clear that a higher voltage usually results in a stronger 
field.

\figOctave{BvasteD}{Magnetic field in function of the bank voltage}

\subsection{Peak field}

In total, we did 28 measurements at maximum capacitor voltage. Of these, 10 
reached magnetic field strengths of over 1 tesla. Because we couldn't research 
every parameter that could change the magnetic field, we didn't find a 
procedure to consistently generate such high fields. Although we lack some 
control, high fields were succesfully achieved in more than 60\% of the 
experiments.

\subsection{Plasma conduction}

A few microseconds after the coil is violently destroyed, a plasma is formed 
because of the strong electric field. This plasma of metal vapour starts to 
conduct a short time later. We have observed this phenomenon a few times and an 
exampe has been plotted in figure \ref{plasma1}. Around zero time the magnetic 
field collapses as the coil disintegrates. The strong back emf turns the 
current around for a very short period. After that, the plasma current starts 
to flow. Notice that the magnetic field doesn't collapse as fast anymore as the 
new current keeps it up. The characteristic kink in the current is present 
again.

\fig{width=0.95\textwidth}{plasmaArc}{Long exposure of the plasma during a 
high energy pulse through an aluminum foil coil. The yellow colour of the 
plasma is partly due to the yellow plastic used to block excess light and 
protect the camera sensor.}

\figOctave{plasma1}{An example of plasma conduction. The magnetic field (blue) 
collapses at 0 as the coil disintegrates. A few moments later, the plasma 
current (green) starts to flow.}

\figOctave{plasma2}{Another example of plasma conduction. The magnetic field 
(blue) is plotted against the time. Note the kink in the current (green).}

\subsection{Choice of material}

Coils made from copper (Cu) and aluminium (Al) were tested. The copper sheet 
was at 0.5\,mm a lot thicker than the aluminium foil of only 15\,$\mu$m.  
Nevertheless, the peak fields achieved with aluminium were as high as they were 
with copper. The experiments with copper were a lot more consistent as it 
seemed the brittle aluminium got tore off by the magnetic stress too soon, 
before peak fields were achieved. Evidence of this is visible in figures 
\ref{plasma1} and \ref{plasma2}: the field was increasing and clearly hadn't 
reached it peak at the moment the coil disintegrated and a plasma was formed.  
Sadly, the plasma never was good enough a conductor or appeared too late for it 
to generate an appreciable field.

% vim: ft=tex
