\section{Results}
After having adapted the apparatus with the extra measuring capabilities, 
we performed some standard measurements of the maximum magnetic field. We 
also used the added $V$ and $I$ readings to XXXXXXXXXXXXXXXXXXXXX.
%XXX plasma stroom, vermogen

Because modifying the apparatus took up quite a bit of time, the 
measurements below are not as thorough as we preferred. We chose to touch 
several small topics that can be investigated further in the future.

\subsection{Magnetic field in function of the width of the coil}
One expects to find an optimal value of the coil width. Too narrow a coil 
will experience a lot of Joule heating because of its high resistance. This 
can cause the metal to melt and evaporate before the maximum field strength 
is reached. A very wide coil will distribute its magnetic energy over a 
large volume, thus loosing efficiency.

The magnetic field of 23 measurements at maximum voltage was recorded. The 
results of these measurements are displayed in figure \ref{BvasteV}. The 
magnetic field is rendered in function of the width of the coil. Although 
we expected there to be an optimal coilwidth, we can discern no correlation 
between the field strength and the dimensions of the generating coil. We 
presume more measurements have to be performed under more strict conditions 
to determine any relation.

\figOctave{BvasteV}{Magnetic field in function of the coildwidth}

\subsection{Magnetic field in function of the bank voltage}

The same 23 measurements are displayed in figure \ref{BvasteD}. This time, the 
magnetic field is rendered in function of the voltage of the capacitor.  
Although no real relationship can be determined, it is clear that a higher 
voltage usually results in a stronger field.

\figOctave{BvasteD}{Magnetic field in function of the bank voltage}

\subsection{Plasma conduction}

A few microseconds after the coil is violently destroyed, a plasma is formed 
because of the strong electric field. This plasma of metal vapour starts to 
conduct a short time later. We have observed this phenomenon a few times and an 
exampe has been plotted in figure \ref{plasma1}. Around zero time the magnetic 
field collapses as the coil disintegrates. The strong back emf turns the 
current around for a very short period. After that, the plasma current starts 
to flow. Notice that the magnetic field doesn't collapse as fast anymore as the 
new current keeps it up. The characteristic kink in the current is present 
again.

XXX PLASMA FOTO XXX

\figOctave{plasma1}{An example of plasma conduction. The magnetic field (blue) 
collapses at 0 as the coil disintegrates. A few moments later, the plasma 
current (green) starts to flow.}

\figOctave{plasma2}{Another example of plasma conduction. The magnetic field 
(blue) is plotted against the time. Note the kink in the current (green).}

\subsection{Choice of material}

Coils made from copper (Cu) and aluminium (Al) were tested. The copper sheet 
was at 0.5\,mm a lot thicker than the aluminium foil of only 15\,$\mu$m.  
Nevertheless, the piek fields achieved with aluminium were as high as they were 
with copper. The experiments with copper were a lot more consistent as it 
seemed the brittle aluminium got tore off by the magnetic stress too soon, 
before piek fields were achieved. Evidence of this is visible in figures 
\ref{plasma1} and \ref{plasma2}: the field was increasing and clearly hadn't 
reached it peak at the moment the coil disintegrated and a plasma was formed.  
Sadly, the plasma never was good enough a conductor or appeared too late for it 
to generate an appreciable field.

% vim: ft=tex
