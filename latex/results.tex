\section{Results}

After having adapted the apparatus with the extra measuring capabilities, 
we performed some standard measurements of the maximum magnetic field. We 
also used the added $V$ and $I$ readings in an attempt to measure a few 
other phenomena such as the current conduction in the plasma after a coil 
breaks, the coil deformation and the power dissipated by the coil and 
plasma.

Because modifying the apparatus took up quite a bit of time, the 
measurements below are not as thorough as we preferred. We chose to touch 
several small topics, some of which are only possible after our 
modifications. These topics can be investigated further and in more depth 
in the future.


\subsection{Magnetic field in function of the width of the coil}

One expects to find an optimal value of the coil width. Too narrow a coil 
will experience a lot of Joule heating because of its high resistance. This 
can cause the metal to melt and evaporate before the maximum field strength 
is reached. A very wide coil will distribute its magnetic energy over a 
large volume, thus loosing efficiency.

The magnetic field of 28 measurements at maximum voltage was recorded. The 
results of these measurements are displayed in figure \ref{BvasteV}. The 
magnetic field is rendered in function of the width of the coil. Although 
we expected there to be an optimal coilwidth, we can discern no correlation 
between the field strength and the dimensions of the generating coil. We 
presume more measurements have to be performed under more strict conditions 
to determine any relation.

\figOctave{BvasteV}{Magnetic field in function of the coildwidth}


\subsection{Magnetic field in function of the bank voltage}

We also resarched the relationship between peak field and voltage of the 
capacitor bank. Measurements with a coil width of 6$\pm$1\,cm were selected and 
rendered in figure \ref{BvasteD}. The peak magnetic field is displayed in 
function of the voltage of the capacitor. Although no real relationship can be 
determined, it is clear that a higher voltage usually results in a stronger 
field.

\figOctave{BvasteD}{Magnetic field in function of the bank voltage}


\subsection{Peak field}

In total, we did 28 measurements at maximum capacitor voltage. Of these, 10 
reached magnetic field strengths of over 1 tesla. Because we couldn't 
sufficiantly research every parameter that could change the magnetic field, 
we didn't find a procedure to consistently generate such high fields.  
Although we lack some control, high fields were succesfully achieved in 
more than 60\% of the experiments with a maximum at 2.7\,T.


\subsection{Choice of material}

Coils made from copper (Cu) and aluminium (Al) were tested. The copper sheet 
was at 0.5\,mm a lot thicker than the aluminium foil of only 15\,$\mu$m.  
Nevertheless, the peak fields achieved with aluminium were as high as they 
were with copper, within the level of statistical uncertainty because of 
our small sample size.

The experiments with copper were a lot more consistent, though, as it 
seemed the brittle aluminium got tore off by the magnetic stress too soon, 
before peak fields were achieved. Evidence of this is visible in figures 
\ref{plasma1} and \ref{plasma2} (page \pageref{plasma1}). The field was 
still increasing and clearly hadn't reached it peak at the moment the coil 
disintegrated and a plasma was formed. Sadly, the plasma never was good 
enough a conductor or appeared too late for it to generate an appreciable 
field.


\subsection{Inductance of the single turn coil and effect on peak 
field\label{secCoilInductance}}

Having the extra voltage and current readings, we can also try finding 
other interesting effects or phenomena. One such thing is that we are now 
capable of finding the inductance of the single turn coil by ftting the 
results to the model above.

Our choice was to fit the current measurement in a similar fashion as we 
did in section \ref{secIcalib}. The difference between the fitted 
inductance and the parasitic inductance of the apparatus found in section 
\ref{secIcalib} should give the inducance of the single turn coil.

However, when plotting and fitting these curves, no significant deviation 
was noted from the curves found in section \ref{secIcalib} (eg. figure 
\ref{fitI800V} on page \pageref{fitI800V}). The small inductance of the 
single turn coil is thus insignificant compared to the large inductance of 
the rest of the apparatus.

This has a few concequences regarding the peak field. For one, the energy 
available in the capacitor gets distributed over the magnetic energy in the 
single turn coil and the inductance of the rest of the circuit. Having such 
high a parasitic inductance severely limits the energy available to the 
coil.

Secondly, having such a high total inductance lengthens the time until the 
peak field is generated. This means the single turn coil has to stand up to 
the large stresses for a longer period of time. This can of course be 
mitigated by choosing a lower capacitance of the capacitor bank. In order 
to store the same energy, however, the voltage of this bank will have to 
rise. There is, however, a practical limit to this voltage \cite{herlach}.  
Voltages of 50\,kV and above become very difficult to handle.

In order to generate higher peak fields, it would be wise to try and limit 
the stray inductance of the circuit. It could also be beneficial to choose 
a capacitor with a smaller capacitance (but higher rated voltage). One 
remark here is that these operations both decrease the pulse duration, so 
that any measurements in the high magnetic field need to be very fast.


\subsection{Plasma conduction}

\fig{width=0.95\textwidth}{plasmaArc}{Long exposure of the plasma during a 
high energy pulse through an aluminum foil coil. The yellow colour of the 
plasma is due to the yellow plastic used to block excess light and protect 
the camera sensor.}

A few microseconds after the coil is violently destroyed, a plasma is 
formed because of the strong electric field as photographed in figures 
\ref{plasmaArc} and \ref{fuckYeah}. This plasma of metal vapour starts to 
conduct a short time later.

We have observed this phenomenon a few times and chose to examine it in 
more detail. The following results are obtained by using an aluminum foil 
coil with a width of approximately 8\,cm.

An exampe has been plotted in figure \ref{plasma1}. Around time zero, the 
magnetic field collapses as the coil disintegrates. The strong back emf 
turns the current around for a very short period. After that, the plasma 
current starts to flow. Notice that the magnetic field doesn't collapse as fast 
anymore as the new current keeps it up. The characteristic kink in the current 
is present. The magnetic field never gets as strong as it was before anymore 
because the resistance of the plasma is a lot higher and it is being blown away 
from the sample space, as can clearly be seen in \ref{fuckYeah}

\fig{width=0.95\textwidth}{fuckYeah}{Long exposure of the plasma during a 
violent high energy pulse through an aluminum foil coil. No background light 
was added, all the light came from the plasma. Notice that the debris violently 
fly away from the coil and thus protect the sample space.}

\figOctave{plasma1}{An example of plasma conduction. The magnetic field 
(blue) collapses slighty after $t=0$ as the coil disintegrates. A few 
moments later, the plasma current (green) starts to flow, restoring some of the 
field. The current is in arbitrary units but the peak current is expected to be 
around 15\,kA}

A similar plot is shown in figure \ref{plasma2}, where the coil 
desintegrates at a field of 1\,T. The current is once again kinked and the 
field follows a similar parttern as before. Note that the coil 
disintegrated well before the maximum field was reached, thus a great deal 
of potential was lost here. Of course, the coil was specially made from 
thin aluminum foil to examine the effect of the plasma, not to generate a 
high magnetic field.

\figOctave{plasma2}{Another example of plasma conduction. The magnetic 
field (blue) is plotted against the time. Note the kink in the current (green).  
The current is in arbitrary units but the peak current is expected to be around 
15\,kA}


\subsection{Power dissipated in the coil and plasma, their resistance and coil 
deformation}

Using the values calibrated in the previous section, one can easily 
calculate the power dissipated or generated in the coil (and later on, in 
the plasma) by multiplying them ($P = VI$).  Doing so presents us with a 
plot like figure \ref{power}. This plot was generated by a full-energy 
pulse through a copper coil of width 4\,mm. The discontinuity around 
100$\mu$s is the moment the coil breaks, after which all current conduction 
occurs through the plasma.

At first sight, these results look sensible. The power dissipation of the 
plasma can be modeled an that of a resistor with a small inductance.  
Indeed, the power dissipated is mostly positive (because of the 
resistance), but has some negative peaks (where some enery is received from 
the magnetic field).

However, when we integrate the power over time to find the total dissipated 
energy, we find that it is a multiple (about 3 times) of the energy stored 
in the capacitor! As this is physically impossible, we have to conclude 
that either the voltage over or the current through the coil is measured 
incorrectly. Since the voltage is measured directly, the current is 
probably not completely right.  Our experience showed that shifting the 
cables around a bit often could change the phase of the measured signal 
varies a considerable amount.  Another problem is that the shunt used 
consists of a piece of wire that has some unnegligible self-inductance, 
meaning that it is not a linear component and thus can't be modeled as 
such.

\figOctave{power}{Power dissipated in the coil and plasma. The coil breaks 
after 100$\mu$s. Quantitatively incorrect because more energy is dissipated 
than is initially available in the capacitor}

Although we can only determine some qualitative behaviour, we're sure that if a 
method is used to determine the current with some degree of confidence, the 
power can certainly be determined as well.

Moreover, having a correct power measurement combined with a 
$\derivs{B}{t}$ reading should allow us to get a rough estimate of the 
deformation of the coil (when this survives multiple periods of the pulse, 
before breaking).

Indeed, the energy in the magnetic field can be crudely approximated by $U_B = 
V B^2/2\mu_0$ (ignoring edge effects).  Thus, power delivered to the coil can 
be linked to the change in energy of the magnetic field. This expected change 
in field strength can be linked to the measured change. Any (downwards) 
deviations of the measured field from the expected field can be linked to a 
volume (increase) of the coil, because $V$ is an explicit fuction of the time 
$t$, meaning that it also plays a role in $\derivs{U_b}{t}$.

This could give some insight to the expansion velocity of the coil.  
However, because the lack of precision measurements, such results are not 
attainable with our setup.


% vim: ft=tex
