\section{Results}

\subsection{Magnetic field in function of the width of the coil}

The magnetic field of 23 measurements at maximum voltage was recorded. The 
results of these measurements are displayed in figure \ref{BvasteV}. The 
magnetic field is rendered in function of the width of the coil. Although we 
expected there to be an optimal coilwidth, there is absolutely no correlation 
between the field strength and the dimensions of the generating coil. We 
presume more measurements have to be performed under more strict conditions to 
determine any relation.

\figOctave{BvasteV}{Magnetic field in function of the coildwidth}

\subsection{Magnetic field in function of the bank voltage}

The same 23 measurements are displayed in figure \ref{BvasteD}. This time, the 
magnetic field is rendered in function of the voltage of the capacitor.  
Although no real relationship can be determined, it is clear that a higher 
voltage usually results in a stronger field.

\figOctave{BvasteD}{Magnetic field in function of the bank voltage}

\subsection{Plasma conduction}

A few microseconds after the coil is violently destroyed, a plasma is formed 
because of the strong electric field. This plasma of metal vapour starts to 
conduct a short time later. We have observed this phenomenon a few times and an 
exampe has been plotted in figure \ref{plasma}. Around zero time the magnetic 
field collapses as the coil disintegrates. The strong back emf turns the 
current around for a very short period. After that, the plasma current starts 
to flow. Notice that the magnetic field doesn't collapse as fast anymore as the 
new current keeps it up.

XXX PLASMA FOTO XXX

\figOctave{plasma1}{An example of plasma conduction. The magnetic field (blue) 
collapses at 0 as the coil disintegrates. A few moments later, the plasma 
current (green) starts to flow.}
