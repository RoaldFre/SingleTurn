
\section{Calibrating the current and fitting system parameters 
\label{secIcalib}}
Sustainably producing currents of the magnitude of a typical pulse is not 
possible, so we cannot apply a known current and measure the voltage over 
the shunt to calibrate the current reading.

We can, however, fit the measured voltage to the expected current in an RLC 
circuit with a scaling factor that needs to be determined. This factor is 
composed of the resistance of the shunt, scaled by the voltage divider. 

The measurement was made without a single turn coil present. The vice was 
just clamped together. Thus, the only inductance that played a role was the 
total stray inductance of the circuit. Several measurements were made with 
different initial voltages on the capacitor.

The function that needs to be fitted is thus
$$
\Vshunt(t - \Delta t) = R_\mathrm s I(t) / \xi
$$
with $\xi = 46.6$ the reduction of our voltage divider and $I(t)$ given by 
equation \ref{current} where $C$ is known to be 400\,$\mu$F and the initial 
voltage over the capacitor $V_0$ is known too.

We also incorporate a time difference to align the graph. The triggering of 
the oscilloscope (the measured point $t = 0$ for $\Vshunt$) surely did not 
happen at the exact moment the current started flowing (the point $t = 0$ 
in the expression for $I(t)$). This $\Delta t$ is an extra factor that 
needs to be fitted.

The remaining factors that need to be fitted are $R_\mathrm s$ the 
resistance of the shunt, $R$, the total resistance of the circuit and $L$, 
the total inductance of the circuit.

The fitting was done with a custom Octave script that does a $\chi^2$ 
minimization. The results we got are tabulated below, where the value of 
$\Delta t$ is omitted because it serves no meaning (it is merely dependent 
on when the osilloscope triggered).
$$
\begin{array}{c|c|c|c}
V_0 (\mathrm V)&
	L (\mu\mathrm H)&
			R (\mathrm m \Omega)&
				\Rshunt (\mathrm m \Omega)\\\hline
71&	0.83 \pm 0.06&	12 \pm 3&	43 \pm 8\\
100&	0.84 \pm 0.03&	8.9 \pm 1.3&	9.0 \pm 1.1\\
200&	0.82 \pm 0.01&	8.3 \pm 0.6&	5.3 \pm 0.3\\
400&	0.82 \pm 0.01&	7.1 \pm 0.6&	2.0 \pm 0.1\\
800&	0.82 \pm 0.01&	7.7 \pm 0.7&	1.12 \pm 0.09\\
\end{array}
$$

We find that the value of $L$ is constant throughout the voltage regime, as 
expected. The resistance of the circuit on the other hand, seems to 
decrease as the pulse becomes stronger. Even more worrying is the fact that 
our $\Rshunt$ exceeds the total resistance. This is likely caused by 
parasitic impedances in the cabling and the measurement circuit. The value 
of $\Rshunt$ should therefore not be looked at as a real resistance, but 
merely as a factor that describes the total effect of various impedances on 
the amplitude of the measured voltage. This is all we need in order to 
calculate a current amplitude from a voltage reading.

WHYYYYYY???

% vim: ft=tex
