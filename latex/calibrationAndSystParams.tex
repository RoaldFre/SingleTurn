
\section{Calibrating the current and fitting system parameters 
\label{secIcalib}}
Sustainably producing currents of the magnitude of a typical pulse is not 
possible, so we cannot apply a known current and measure the voltage over 
the shunt to calibrate the current reading.

We can, however, fit the measured voltage to the expected current in an RLC 
circuit with a scaling factor that needs to be determined. This factor is 
composed of the resistance of the shunt, scaled by the voltage divider. 

The measurement was made without a single turn coil present. The vice was 
just clamped together. The only inductance that played a role was the total 
stray inductance of the circuit. The capacitor was charged to 346\,V.

The function that needs to be fitted is thus
$$
\Vshunt(t - \Delta t) = R_\mathrm s I(t) / \xi
$$
with $\xi = 46.6$ the reduction of our voltage divider and $I(t)$ given by 
equation \ref{current} where $C$ is known to be 400\,$\mu$F and the value 
of $V_0$ is known too (346\,V for the particular calibration measurement).

We also incorporate a time difference to align the graph. The triggering of 
the oscilloscope (the measured point $t = 0$ for $\Vshunt$) surely did not 
happen at the exact moment the current started flowing (the point $t = 0$ 
in the expression for $I(t)$). This $\Delta t$ is an extra factor that 
needs to be fitted.

The remaining factors that need to be fitted are $R_\mathrm s$ the 
resistance of the shunt, $R$, the total resistance of the circuit and $L$, 
the total inductance of the circuit.

The fitting was done with a custom Octave script that does a $\chi^2$ 
minimization. The results we got are
$$
L = (212 \pm 6)\,\mathrm{nH}
\qquad
R = (5.2 \pm 0.7)\,\mathrm{m}\Omega
\qquad
\Rshunt = (1.5 \pm 0.2)\,\mathrm{m}\Omega
$$
with a $\Delta t$ of $(-2.4 \pm 0.9)\mu\mathrm s$.






% vim: ft=tex
