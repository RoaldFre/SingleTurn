\section{Modifications}
We modified the original apparatus with extra measurement capabilities and 
gave it general revision and cleaning.

\subsection{Maintenance}
The apparatus already had a history of many high energy pulses, which meant 
it was time for some cleaning and maintenance. The plates of the spring 
loaded switch had gotten charred and rough from the frequent arcing when 
commuting the high currents. They were filed smooth and flush 
again\footnote{Interesting fact, the first few high current pulses after 
smoothing out the switch actually welded it shut. Both plates made contact 
within a very small tolerance and a large surface area stuck together.}.

The vice that holds the single turn coil also needed to get filed down, as 
it had gotten rough and showed several bumps, decreasing the effective 
contact area.

\subsection{Replacement of pick up coil}
During the maintenance and adding the extra measurement connectors, one of 
the brittle wires of the pick up coil got damaged beyond repair. This was 
the ideal opportunity to wind a new pick up coil that could be calibrated. 
The new coil was wound from a slightly thicker and less brittle copper 
wire. It has a diameter of about 4\,mm and consists of 10 turns of wire. 
The total equivalent calibrated surface area reads 136\,mm$^2$.

\subsection{Measuring current through and voltage over the coil}
The voltage over the coil can be measured straightforward by measuring over 
both sides of the clamping vice. The current measurement is more difficult.  
We opted for a simple and non-intrusive method. We measure the current by 
measuring the voltage drop over a connecting wire that acts as a shunt 
resistor.

\subsubsection{Choosing measuring points}
Care needs to be taken that both probe connections share the same ground.  
If they don't, the oscilloscope that is used to read out the signal will 
short out both grounds anyway. This will disturb the measurement and can 
potentially send very large currents through the scope.

Practically, the easy and tidy option was to choose both measureing points 
as depicted in figure \ref{circuit-old}. The voltage over the coil is given 
by $V_1$ and the current is proportional to (the negative of) $V_2$.

\fig[htb]{width=0.55\textwidth}{circuit-old}{Initial setup of measuring 
probes}

However, as will be shown below [XXX section XXX], the (stray) inductance 
of the connecting wires is much greater than the inductance of the single 
turn coil. This means that the wire coming from the capacitor and leading 
to the coil, and the wire from the switch to the coil both have a much 
higher effect on the measured voltage than the single turn coil itself.  
The resistance of the wires (which we actually use as a shunt for the 
current reading!) also adds to the measured voltage.

This effect skewed the measurements and was mitigated by choosing a 
different placement of the probe connections. This setup is depicted in 
figure \ref{circuit-new}.

\fig[htb]{width=0.55\textwidth}{circuit-new}{Improved setup of measuring 
probes}

The voltage $V_1$ over the coil is measured with as little extra wire in 
between as possible. The voltage $V_2$ is measured over a wire that is as 
straight as possible, to minimize the inductance. This setup offered more 
reliable and correct measurements than the previous one.
%XXX te vaag XXX

\subsubsection{Getting the voltage down}
Because the capacitor gets charged to 850\,V for the high energy pulses, 
the voltage over the measuring points can get very high. In order to safely 
measure this voltage, a custom voltage divider (figure \ref{Vdiv}) was 
added for both measuring points.

\fig[htb]{width=0.35\textwidth}{Vdiv}{Voltage divider}

%XXX correct ??? XXX
At first, we used values of $R_1 = 120\,\mathrm{k}\Omega CORRECT?$ and $R_2 
= 15\,\mathrm{M}\Omega CORRECT?$, leading to a 126$\times$ reduction in the 
measured voltage.

At the highest possible voltage of 850\,V, these resistors would dissipate 
about 0.06\,W.

However, we noted some dampening effects when measuring via these voltage 
dividers compared to a direct measurement. We assume the internal 
capacitance (and in general the impedance) of the connecting coax cables 
was a bit too high for the given frequency and the (high) output impedance 
of the voltage divider.

%XXX grafiek >_< XXX

We therefore choose to lower the values to $R_1 = 18\,\mathrm{k}\Omega?$ 
and $R_2 = 820\,\mathrm{k}\Omega$. This gives a power dissipation of 1\,W 
at peak voltage, still much lower than the power dissipated in the coil.






The input impedance of the used oscilloscope was 1\,M$\Omega$.


we measure needs
voltage divider

\subsubsection{Effect of stray inductance on the measured current}

\subsubsection{Calibrating the current}

\subsubsection{Effect of coax connector}
impedance of voltage divider still too high?

EERST MOTIVATIE UITLEGGEN, DIT IS ECHT WERKELIJKE BESCHRIJVING WAT EN HOE


