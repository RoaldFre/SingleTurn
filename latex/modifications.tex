\section{Modifications}
We modified the original apparatus with extra measurement capabilities and 
gave it general revision and cleaning.

\subsection{Maintenance}
The apparatus already had a history of many high energy pulses, which meant 
it was time for some cleaning and maintenance. The plates of the spring 
loaded switch had gotten charred and rough from the frequent arcing when 
commuting the high currents. They were filed smooth and flush 
again\footnote{Interesting fact, the first few high current pulses after 
smoothing out the switch actually welded it shut. Both plates made contact 
within a very small tolerance and a large surface area stuck together.}.

The vise that holds the single turn coil also needed to get filed down, as 
it had gotten rough and showed several bumps, decreasing the effective 
contact area.

\subsection{Replacement of pick up coil}
During the maintenance and adding the extra measurement connectors, one of 
the brittle wires of the pick up coil got damaged beyond repair. This was 
the ideal opportunity to wind a new pick up coil that could be calibrated. 
The new coil was wound from a slightly thicker and less brittle copper 
wire. It has a diameter of about 4\,mm and consists of 10 turns of wire. 
The total equivalent calibrated surface area reads 136\,mm$^2$.

\subsection{Measuring current through and voltage over the coil}


EERST MOTIVATIE UITLEGGEN, DIT IS ECHT WERKELIJKE BESCHRIJVING WAT EN HOE


\fig[htb]{width=0.55\textwidth}{circuit-old}{First setup of measuring probes}
\fig[htb]{width=0.55\textwidth}{circuit-new}{Improved setup of measuring probes}
