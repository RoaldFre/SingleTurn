\section{Generation of magnetic fields}

Magnetic fields are of the utmost importance to the experimental physicist.  
They are mostly used in conjunction with cryostats to research the properties 
of various solids. For this reason, various techniques have been invented to 
generate strong magnetic fields. There are basically three classes of 
techniques: continuous, pulsed non-destructive and pulsed destructive field 
generation methods. Various techniques have been summed up in table 
\ref{methodTable} together with their strongest produced fields 
\cite{highPulsedMagnet} \cite{recordField}.

\begin{table}
\caption{Various magnetic field generation methods}
\label{methodTable}
\begin{tabular}{l|l}
Method & Field (T) \\
\hline
\hline
Permanent magnet & 1.25\\
Resistive electromagnet & 36.2\\
Hybrid electromagnet & 45\\
Explosive & 2800 \\
\end{tabular}
\end{table}

Iron core electromagnets can only be used for fields without saturating up to 
2.16 Tesla \cite{ironSaturation}. For this reason, most coils are used without 
a core. 

\section{RLC circuit theory}

The circuit containing the coil and the capacitator can be crudely modeled as a 
RLC circuit. The value of $C$, the capacitance, was determined to be $400\,\mu 
F$ with a multimeter, matching the given specification. The values for $R$ and 
$L$, respectively the resistance and the self-inductance of the circuit are 
nigh impossible to measure, as they vary considerably from values that can be 
measured with our available measuring instruments. This is mainly because of 
the high applied voltages and currents and the frequencies at which we work, 
and the relatively low values of both values.

Preliminary measurements showed that the system is very underdamped ($\zeta = 
0.1$). The equation for the damped harmonic oscillator goes as follows
$$
Q(t) = V_0 C e^{-\zeta \omega t} \left( \cos{\omega_d t} + \frac{\zeta} 
{1-\zeta^2} \sin{\omega_d t}\right)
$$
Where $Q$ is the charge in the capacitor in function of time, $V_0$ is the 
initial voltage, $C$ is the capacitance, $\omega$ the undamped natural 
frequency, $\zeta$ the damping factor and $\omega_d$ the damped natural 
frequency defined as such

$$
\omega^2 = \frac{1}{LC} \qquad
\zeta = \frac{R}{2L \omega} \qquad
\omega_d = \omega \sqrt{1-\zeta^2}
$$

Differentiating this expression, one obtains the current

$$
I(t) = \deriv{Q}{t} = -V_0 C \frac{\omega}{\sqrt{1-\zeta^2}} e^{-\zeta \omega 
t} \sin{\omega_d t}
$$

Using this expressions and the measured voltage over a shunt, we were able to 
determine with some precision the effective self-inductance and resistance of 
the circuit.
