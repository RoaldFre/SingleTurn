\section{Generation of magnetic fields}
\subsection{History}

Magnetic fields are of the utmost importance to the experimental physicist.  
For example, they are widely used to resarch the proporties of various 
solids, often in conjuction with cryostatic temperatures.

Thus, various techniques have been invented to generate strong magnetic 
fields.  There are basically three classes of techniques: continuous, 
pulsed non-destructive and pulsed destructive field generation methods.  
Various techniques have been summed up in table \ref{methodTable} together 
with their strongest produced fields \cite{highPulsedMagnet} 
\cite{recordField}.

Iron core electromagnets can only be used for fields up to 2 tesla, because the 
material saturates at these flux densities. For this reason, most coils are 
used without a core. The strongest permanent magnetic fields are generated 
using superconducting electromagnets or a hybrid of both resistive and 
superconductive magnets. When even higher fields are needed, 
superconductors become unavailable because their conductive properties are 
lost as the field crosses a critical value. Thus, to deal with the 
extremely high heat that is generated, pulsed fields are used. Valuable 
data ought to be collected in the small moment the magnetic field is 
present. \cite{herlach} 

\begin{table}
\begin{center}
\caption{Various magnetic field generation methods and their maximum 
attainable field}
\label{methodTable}
\begin{tabular}{c|c}
Method & Field (T) \\
\hline
Permanent magnet & 1.3\\
Resistive electromagnet & 36\\
Hybrid electromagnet & 45\\
Pulsed (non-destructively) & 89\\
Explosive & 2800 \\
\end{tabular}
\end{center}
\end{table}

\subsection{Single turn coil}

The method for generating magnetic fields we researched is the ``single 
turn coil'' method. As the name says, the coils used consist of a single 
loop of metal. A capacitor (bank) is used to store electrical energy which 
is released into the circuit. When operated at the ideal power level, Joule 
heating and magnetic stresses in the coil will cause it to disintegrate 
shortly after the peak field is reached. The metal of which the coil is 
composed evaporates  and after a short pause, the plasma of metal vapour 
starts to conduct again. During the short period where coil stays more or 
less together, a very high field is generated in the measurement 
``chamber''. 

The single turn coil method is unique among the destructive methods as 
there isn't any destruction in the sample space. After the violent 
explosion of the coil, the material is driven outwards because of the 
Maxwell stress, directed entirely away from the sample space 
\cite{herlachArticle}.  This makes repeated measurements possible as well 
as the possibility of using cryostats. The expelled coil fragments move at 
dangerously high speeds and care must be taken to protect apparatuses and 
personnel \cite{singleTurn}. During the experiments we performed, about 50 
shots were fired. The pickup-coil was protected by a thin sheet of Kapton 
that also served to insulate the coil. The sample space and coil suffered 
no damage at all.

\subsection{RLC circuit theory}

The circuit containing the coil and the capacitator can be approximated as an 
RLC circuit. A more in depth analysis would require solving the non-linear 
Maxwell equations, for example by finite-element 
analysis.\cite{herlachArticle}

The value of $C$, the capacitance, was determined to be $400\,\mu F$ with a 
multimeter, matching the given specification. The values for $R$ and $L$, 
respectively the resistance and the self-inductance of the circuit are very low 
and nigh impossible to measure directly. These values and will be computed in 
section \ref{secIcalib} by fitting the model below to the measured data.

Preliminary measurements showed that the system is very underdamped ($\zeta 
\approx 0.1$), so we don't need to consider the critically and the 
overdamped case.  The equation for the damped harmonic oscillator goes as 
follows \cite{serway}
\begin{equation}
Q(t) = V_0 C e^{-\zeta \omega t} \left( \cos{\omega_d t} + \frac{\zeta} 
{1-\zeta^2} \sin{\omega_d t}\right)
\end{equation}
Where $Q$ is the charge in the capacitor in function of time, $V_0$ is the 
initial voltage, $C$ is the capacitance, $\omega$ the undamped natural 
frequency, $\zeta$ the damping factor and $\omega_d$ the damped natural 
frequency defined as
$$
\omega^2 = \frac{1}{LC} \qquad
\zeta = \frac{R}{2L \omega} \qquad
\omega_d = \omega \sqrt{1-\zeta^2}
$$

Differentiating this expression, one obtains the current
\begin{equation}
\label{current}
I(t) = \deriv{Q}{t} = -V_0 C \frac{\omega}{\sqrt{1-\zeta^2}} e^{-\zeta \omega 
t} \sin{\omega_d t}
\end{equation}

Using this expressions and the measured voltage over a shunt, we were able to 
determine with some precision the effective self-inductance and resistance 
of the circuit in section \ref{secIcalib}.
