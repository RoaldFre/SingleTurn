\section{Generation of magnetic fields}
\subsection{History}

Magnetic fields are of the utmost importance to the experimental physicist.  
They are mostly used in conjunction with cryostats to research the properties 
of various solids. For this reason, various techniques have been invented to 
generate strong magnetic fields. There are basically three classes of 
techniques: continuous, pulsed non-destructive and pulsed destructive field 
generation methods. Various techniques have been summed up in table 
\ref{methodTable} together with their strongest produced fields 
\cite{highPulsedMagnet} \cite{recordField}.

Iron core electromagnets can only be used for fields up to 2 tesla, because the 
material saturates at these flux densities. For this reason, most coils are 
used without a core. The strongest permanent magnetic fields are generated 
using superconducting electromagnets or a hybrid of both resistive and 
superconductive magnets. As higher and higher fields are generated, 
superconductors become unavailable because their conductive properties are lost 
as the field crosses a critical value. Thus, to deal with the extremely high 
heat that is generated, pulsed fields are used. Valuable data ought to be 
collected in the small moment the magnetic field is present. \cite{herlach} 

\begin{table}
\begin{center}
\caption{Various magnetic field generation methods}
\label{methodTable}
\begin{tabular}{c|c}
Method & Field (T) \\
\hline
Permanent magnet & 1.25\\
Resistive electromagnet & 36.2\\
Hybrid electromagnet & 45\\
Pulsed (non-destructively) & 89\\
Explosive & 2800 \\
\end{tabular}
\end{center}
\end{table}

\subsection{Single turn coil}

The method for generating magnetic fields we researched is the ''single turn 
coil'' method. As the name says, the coils used consist of a single loop of 
metal. A capacitor (bank) is used to store electrical power which is released 
into the circuit. The metal of which the coil is composed evaporates after a 
few microseconds and after a short pause, the plasma of metal vapour starts to 
conduct again. During the short period where coil stays more or less together, 
a very high field is generated in the measurement ''chamber''. 

The single turn coil method is unique among the destructive methods there isn't 
any destruction in the sample space. After the violent explosion of the coil, 
the material is driven outwards, directed entirely away from the sample space.  
This makes repeated measurements possible as well as the possibility of using 
cryostats. The expelled coil fragments move at dangerously high speeds and care 
must be taken to protect apparatuses and personnel. \cite{singleTurn} 
\cite{herlachArticle} During the experiments we performed, about 50 shots were 
fired. The pickup-coil was protected by a thin sheet of Kapton that also served 
as to insulate the coil. The sample space and coil suffered no damage at all.

\subsection{RLC circuit theory}

The circuit containing the coil and the capacitator can be crudely modeled as a 
RLC circuit. The value of $C$, the capacitance, was determined to be $400\,\mu 
F$ with a multimeter, matching the given specification. The values for $R$ and 
$L$, respectively the resistance and the self-inductance of the circuit are 
nigh impossible to measure, as they vary considerably from values that can be 
measured with our available measuring instruments. This is mainly because of 
the high applied voltages and currents and the frequencies at which we work, 
and the relatively low values of both values.

Preliminary measurements showed that the system is very underdamped ($\zeta = 
0.1$). The equation for the damped harmonic oscillator goes as follows 
\cite{serway}
$$
Q(t) = V_0 C e^{-\zeta \omega t} \left( \cos{\omega_d t} + \frac{\zeta} 
{1-\zeta^2} \sin{\omega_d t}\right)
$$
Where $Q$ is the charge in the capacitor in function of time, $V_0$ is the 
initial voltage, $C$ is the capacitance, $\omega$ the undamped natural 
frequency, $\zeta$ the damping factor and $\omega_d$ the damped natural 
frequency defined as such

$$
\omega^2 = \frac{1}{LC} \qquad
\zeta = \frac{R}{2L \omega} \qquad
\omega_d = \omega \sqrt{1-\zeta^2}
$$

Differentiating this expression, one obtains the current

$$
\label{current}
I(t) = \deriv{Q}{t} = -V_0 C \frac{\omega}{\sqrt{1-\zeta^2}} e^{-\zeta \omega 
t} \sin{\omega_d t}
$$

Using this expressions and the measured voltage over a shunt, we were able to 
determine with some precision the effective self-inductance and resistance of 
the circuit.
