\section{Apparatus}
The apparatus consists of a power supply that charges a large capacitor. 
The energy in this capacitor gets dumped in the single turn coil through a 
spring loaded, high current switch. This coil is mounted over a testsample.

\fig[htb]{width=0.55\textwidth}{circuit}{Overview of the apparatus}

\subsection{Power supply}
The apparatus has a built-in 12\,V 1.2\,Ah sealed lead acid battery to 
provide a portable power supply to charge the battery. Because of the age 
of the apparatus, however, the battery had gone dead and the power was 
provided by a 12V lab power supply.

An inverter boosts the 12V to around 850V to charge the high voltage 
capacitor.

\subsection{Capacitor}
The capacitor used is a 850\,V 400\,$\mu$F bipolar capacitor that can store 
around 150\,J of energy.

\subsection{Switch}
Because of the very high current surges, it is unfeasable to use a standard 
switch to commute this current. Instead, a custom made spring loaded switch 
	is used.

This switch consists of two copper bars, separated by a small amount. 
Suspended above these bars, there is a copper disk. This disk is mounted on 
a notched, spring loaded shaft. The notch slides over a small retaining 
screw. The switch is armed by pulling the shaft out and twisting it, so the 
screw is no longer aligned with the notch.

When the knob is then turned again, the notch lines up with the retaining 
screw and the spring pushes the shaft down. The copper disk is propelled 
towards the copper bars and upon impact the current can flow.

\subsection{Coil}
The apparatus is equipped with a (conducting) vise in which a variety of 
coils can be mounted. 

These coils are usually made of thin metal foil or sheeting. Most of our 
tests are conducted with coils of 0.5\.mm thick copper foil.

The coil is pressed firmly in the vise, with a piece of electrically 
isolating kapton foil inserted in the middle to avoid short circuiting the 
coil.

\subsection{Sample}
In our case, the sample merely consisted of a small pick up coil to 
determine the generated magnetic field. This pick up coil is a 10 turn coil 
with a diameter of about 4\,mm and a calibrated equivalent surface area of 
136\,mm$^2$
