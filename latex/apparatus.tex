\section{Apparatus}

\fig{width=0.95\textwidth}{apparatus}{Overview of the apparatus}

A photograph of the apparatus is given in figure \ref{apparatus}. Its 
internals are sketched in figure \ref{circuit}. It consists of a power 
supply that charges a large capacitor (top).  The energy in this capacitor 
gets dumped in the single turn coil (right) through a spring loaded, high 
current switch (left). The coil is mounted over the testsample.

\fig[htb]{width=0.55\textwidth}{circuit}{Schematic overview of the 
apparatus}


\subsection{Power supply}
The apparatus has a built-in 12\,V 1.2\,Ah sealed lead acid battery to provide 
a portable power supply to charge the capacitor. Because of the age of the 
apparatus, however, the battery had gone dead and the power was provided by a 
12V lab power supply. An inverter boosts the 12V to around 870V to charge the 
high voltage capacitor.


\subsection{Capacitor}
The capacitor used is a 850\,V 400\,$\mu$F bipolar capacitor that can store 
around 150\,J of energy.


\subsection{Switch}
Because of the very high current surges, it is unfeasible to use a standard 
switch to commute this current. Instead, a custom made spring loaded switch is 
used.

This switch consists of two copper bars, separated by a small distance.  
Suspended above these bars is a copper disk, which is mounted on a notched, 
spring loaded shaft. The notch slides over a small retaining screw.  The switch 
is armed by pulling the shaft out and twisting it, so the screw is no longer 
aligned with the notch.

When the knob is then turned again, the notch lines up with the retaining 
screw and the spring pushes the shaft down. The copper disk is propelled 
towards the copper bars and upon impact the current can flow.

\fig{width=0.8\textwidth}{switch}{The spring loaded, high current switch}

A photograph of the switch is shown in figure \ref{switch}. The circular 
copper disk is clearly visible. The switch is armed in this photograph, 
ready to shoot downwards when the knob is turned to the right.


\subsection{Coil}
The apparatus is equipped with a (conducting) vice in which a variety of 
coils can be mounted. These coils are usually made of thin metal foil or 
sheeting.  Most of our tests are conducted with coils of 0.5\,mm thick 
copper foil.

\fig{width=0.8\textwidth}{coils}{Some of the coils used. Copper on the left, 
aluminium on the lower-right, nickel on the upper-right}

An hourglass shape, as seen in figure \ref{coils}, is cut from the sheet 
and bent into the required form.  The feeder plates of the coil preferably 
have a large surface area for the high current to pass through. To ensure 
good conduction, the plates should also be sufficiently smooth and 
parallel. To prepare for a measurement, the coil is pressed firmly in the 
vice.

A piece of Kapton is inserted between the inside of the plates to 
electrically insulate them from one another and protect the sample space 
from any stray fragments or vapour.

\fig{width=0.9\textwidth}{unclampedAndClamped}{An unclamped and clamped coil}

Additionally, a small plastic clamp is used to pinch keep the coil 
together, as seen in figure \ref{unclampedAndClamped}. This increases the 
field by forcing the current to go completely round the sample. It also 
helps the coil withstand the high Maxwell stress that is exerted on the 
material.

We experienced that coils made from weaker materials, like aluminium foil, 
can get ripped apart by the magnetic forces before a considerable field has 
formed.  Finally, a protective plastic casing is installed to protect 
ourselves and the instruments from flying debris.

\subsection{Sample}
In our case, the sample merely consisted of a small pick up coil to 
determine the generated magnetic field. This pick up coil is a 10 turn coil 
with a diameter of about 4\,mm and a calibrated equivalent surface area of 
136\,mm$^2$.

% vim: ft=tex
